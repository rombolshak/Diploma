\chapter*{Заключение}				% Заголовок
\addcontentsline{toc}{likechapter}{Заключение}	% Добавляем его в оглавление
Таким образом, исследована секретность релятивистского протокола квантовой криптографии, обеспечивающего безусловную секретность в условиях потерь в линии связи и неоднофотонности источника.
Этот протокол максимально использует фундаментальные ограничения, диктуемые законами природы, на различимость квантовых состояний и учитывает ограничения специальной теории относительности на максимальную скорость движения элементарных частиц. Данный протокол может быть использован для передачи ключей через открытое пространство как между наземными объектами, так и через низкоорбитальные спутники. Двухпроходность схемы обеспечивает большую стабильность ее работы по сравнению с однопроходными схемами.

Второе, рассмотрен и проанализирован один из протоколов коррекции ошибок, который в настоящее время является стандартом в квантовом распределении ключей, так как раскрывает меньше информации о секретном ключе, чем другие протоколы.

Третье, что является основным результатом данной дипломной работы, разработаны программы, моделирующие и визуализирующие процессы:
\begin{itemize}
  \item распределения ключей по релятивистскому протоколу с имитацией атак подслушивателя и последующим детектированием возникающих из-за этого задержек,
  \item коррекции ошибок по протоколу Cascade.
\end{itemize}
Исходные коды обеих программ доступны онлайн по ссылкам \url{https://github.com/rombolshak/Requc/tree/master/Requc} и \url{https://github.com/rombolshak/Requc/tree/master/Cascade} соответственно.
На прилагаемом к данной работе диске также содержатся исходные коды указанных программ по состоянию на 14 мая 2015 года, а также скомпилированные исполняемые и установочные файлы.


