\section{Основные понятия квантовой теории информации}
\subsection{Квантовые состояния}
При проведении первых опытов над элементарными частицами было обнаружено, что их поведение очень сложно увязать с имевшимися на тот момент представлениями о физических явлениях. Это привело к тому, что после формулировки новых законов, описывающих поведение элементарных частиц, эту часть физики стали называть квантовой теорией, а сложившуюся на тот момент физическую картину мира~--- классической.

\subsubsection{Волновая функция и чистые состояния}
Одно из главных отличий квантовой теории от классической проявляется в самом определении квантовой частицы и её состояния.
Представление о квантовой частице, как о некотором теле, имеющем определенные физические характеристики вроде координаты, размера или массы, оказалось в корне неверным, так как для некоторых частиц не удавалось даже понять, в какой точке пространства они в принципе находятся. Зато оказалось возможным предсказать, как эти частицы будут себя вести.
Трудность заключалась в том, что объяснить поведение частиц удалось только после окончательного отказа от попыток вычислить <<традиционные>> характеристики системы. Это привело к тому, что состояние элементарных частиц и их систем стали представлять с помощью <<волновой функции>>.

Введем понятие \textit{чистого квантового состояния}. 
Таким состоянием будем называть вектор в гильбертовом пространстве $\mathcal H$ с единичной нормой.
Под нормой вектора понимается корень его скалярного квадрата.

Будем обозначать вектор состояния, соответствующий состоянию $\psi$, как \state{\psi}. 
Сопряжённый вектор, соответствующий состоянию $\psi$, будем обозначать как \conjstate{\psi}. 
Скалярное произведение векторов \state{\psi}  и \conjstate{\phi}  будем обозначать как $\left\langle\phi|\psi\right\rangle$, 
а образ вектора \state{\psi} под действием оператора $\mathcal F$ будем обозначать $\mathcal F\state{\psi}$. 
Подобные обозначения в целом согласуются с обозначениями обычной линейной алгебры, но более удобны в квантовой механике, так как позволяют более наглядно и коротко называть используемые векторы.

Если мы рассмотрим два различных состояния, то суперпозиции (всевозможные линейные комбинации) пары соответствующих им векторов дадут двумерное линейное комплексное пространство.
При рассмотрении квантовой системы, состоящей из двух подсистем, пространство состояний строится в виде тензорного произведения. 

Для каждого чистого квантового состояния \state{\psi} можно определить соответствующий ему оператор
$\rho_\psi = \state{\psi} \conjstate{\psi}$, называемый \textit{оператором плотности}.
Этот оператор имеет единичный след, ранг 1 и действует как проектор на чистое состояние \state{\psi}.

\subsubsection{Смешанные состояния}
С помощью операторов плотности вводится общее понятие квантового состояния. \textit{Смешанным квантовым состоянием} называется статистическая смесь нескольких чистых состояний:
\begin{equation}
  \rho = \sum_i p_i \state{\psi_i}\conjstate{\psi_i}, \quad p_i \ge 0 \forall i, \quad \sum_i p_i = 1.
\end{equation}

Очевидно, что след смешанного состояния равен единице. Также несложно показать его положительную определенность:
\begin{equation}\label{positive_defined_operator}
  \conjstate{\varphi}\rho\state{\varphi} = \sum_i p_i |\scalar{\varphi}{\psi_i}|^2 \ge 0 \ \forall \state{\varphi} \in \mathcal{H}.
\end{equation}

Как известно, любой эрмитов оператор $A$ имеет спектральное разложение
\begin{equation}
  A = \sum_i \lambda_i \state{\lambda_i}\conjstate{\lambda_i},
\end{equation}
где собственные значения $\lambda_i$ вещественны, а собственные векторы \state{\lambda_i} ортогональны и нормированны. Это означает, что любой положительный эрмитов оператор с единичным следом можно назвать оператором плотности некоторого квантового состояния: из положительной определенности (\ref{positive_defined_operator}) следует положительность всех собственных значений (которые интерпретируются как вероятностные веса), а из условия единичного следа~--- то, что сумма собственных значений равна единице. В итоге это значит, что такая их комбинация может трактоваться как статистическая смесь, что приводит в общему определению квантового состояния.
\begin{definition}
  Квантовое состояние~--- положительный эрмитов оператор в гильбертовом пространстве с единичным следом.
\end{definition}

Квантовые состояния образуют выпуклое множество $\mathcal{S}(\mathcal{H})$ в пространстве операторов $\mathcal{H}$. Крайними точками этого множества являются чистые состояния, описывающиеся операторами ранга 1.

\subsubsection{Изменение состояний во времени}
Одним из ключевых законов квантовой механики является уравнение Шрёдингера, описывающее изменение квантовых состояний во времени. Традиционно это уравнение записывается как 
\begin{equation}\label{shroudinger_equation}
  i \hbar \frac{d \state{\psi}}{dt} = H \state{\psi}, 
\end{equation}

где $\hbar$~--- постоянная Планка. Эрмитов оператор $H$ называется гамильтонианом системы, и именно он оказывает влияние на её эволюцию.

Так как существует соответствие между унитарными и эрмитовыми операторами \cite{nielsen_chuang}:
\begin{equation} U = e^{iH}, \end{equation}
то уравнение (\ref{shroudinger_equation}) может быть переписано следующим образом:
\begin{equation} \state{\psi'} = U \state{\psi}.\end{equation}

Такой вид оказывается более удобным, так как он означает, что любая эволюция квантовой системы может быть представлена как действие некоторого унитарного преобразования.

\subsubsection{Принцип суперпозиции квантовых состояний}
Квантовая суперпозиция~--- это суперпозиция состояний, которые не могут быть реализованы одновременно с классической точки зрения, это суперпозиция альтернативных (взаимоисключающих) состояний.

Если функции  $\Psi_1$ и $\Psi_2$ являются допустимыми волновыми функциями, описывающими состояние квантовой системы, то их линейная суперпозиция, $\Psi_3 = c_1\Psi_1 + c_2\Psi_2$, также описывает какое-то состояние данной системы. 
Если измерение какой-либо физической величины  $\hat f$ в состоянии  $|\Psi_1\rangle$ приводит к определённому результату $f_1$, а в состоянии  $|\Psi_2\rangle$~-- к результату $f_2$, то измерение в состоянии $|\Psi_3\rangle$ приведёт к результату $f_1$ или $f_2$ с вероятностями $|c_1|^2$ и $ |c_2|^2$ соответственно.

\subsubsection{Кубиты}
Простейшим примером нетривиального квантового объекта является система с двумя базисными состояниями. 
Физическими примерами таких систем могут быть фотоны с соответствующими направлениями поляризации или направления спина электрона.
В этом случае соответствующее гильбертово пространство будет двумерным, его обозначают $\mathcal H^2$.
Если не важна конкретная физическая природа двухуровневой системы, её состояния обозначают как 
\state{0} и \state{1}. Такую систему называют \textit{кубитом} по аналогии с классическим битом.

Произвольное чистое состояние кубита можно записать как
\begin{equation} \state{\psi} = \cos \alpha \state{0} + \sin \alpha \state{1}. \end{equation}

\subsection{Измерения}
Именно процедура измерений квантовых состояний отличает квантовый случай проведения опытов от классического и даёт возможность применения квантовой криптографии. Важнейшим отличием квантовой механики от классической является тот факт, что в общем случае \textit{измерение квантовой системы меняет её исходное состояние}.

\subsubsection{Квантовые наблюдаемые}
В любом эксперименте можно выделить две его стадии: приготовление состояния $\rho$ и его измерение $M$. Измерение не обязано давать точно предсказуемый результат: в общем случае результат измерения~--- это статистический набор исходов $\{x\}$ с соответствующими вероятностями $\mu_\rho(x)$. Естественно требовать, чтобы для статистических ансамблей квантовых состояний результаты их наблюдения также были бы статистическими смесями результатов наблюдения соответствующих отдельных состояний ансамбля. Такое требование называется требованием аффинности:
\begin{equation} \mu_\rho(x) = \sum_i p_i \mu_{\rho_i}(x),\quad \rho = \sum_i p_i \rho_i, \end{equation}
где $p_i$~--- вероятности, с которыми каждое состояние входит в ансамбль состояний.
Этого требования достаточно для следующего утверждения \cite{holevo}.
\begin{theorem}
  Пусть $\rho \rightarrow \mu_\rho$~--- аффинное отображение множества квантовых состояний в вероятностные распределения на конечном множестве $X$. Тогда существует семейство эрмитовых операторов $\{M_x\}$ такое, что
  \begin{equation} M_x \ge 0,\quad \sum_{x \in X} M_x = I,\quad \mu_\rho(x) = \tr \rho M_x. \end{equation}
\end{theorem}

Эта теорема говорит о том, что измерение квантовой системы можно связать с набором положительных эрмитовых операторов, сумма которых равна единичному оператору. В этом случае вероятность каждого из исходов равна следу произведения состояния и оператора, соответствующего данному исходу. Это приводит к определению квантовой наблюдаемой.
\begin{definition}
  Квантовая наблюдаемая со значениями из множества $X$~--- набор эрмитовых операторов $\{M_x\}_{x \in X}$ таких, что
  \begin{equation} M_x \ge 0,\quad \sum_{x \in X} M_x = I. \end{equation}
\end{definition}

Такой набор операторов называют разложением единицы.

Из теоремы следует, что при измерении состояния $\rho$, описываемого разложением единицы $\{M_x\}$, вероятность получить каждый из исходов $x$ равна
\begin{equation} \pr(x|\rho) = \tr M_x \rho, \end{equation}
а для чистого состояния \state{\psi} в силу свойств следа эта вероятность выражается более просто:
\begin{equation} \pr(x|\rho_\psi) = \conjstate{\psi} M_x \state{\psi}. \end{equation}

\subsubsection{Коллапс волновой функции}
Важным законом квантовой механики является коллапс волновой функции, или редукция. Это свойство означает переход состояния после измерения в одно из собственных состояний оператора измерения. Так, при измерении $\{M_i\}$ и получении результата $i$ исходное состояние будет преобразовано в 
\begin{equation} \rho'_i = \frac{\sqrt{M_i} \rho \sqrt{M_i}}{\tr M_i \rho}. \end{equation}

Это одно из важнейших для квантовой криптографии свойств, поскольку оно говорит о том, что попытки измерить систему ведут к помехам. Из этого следует, что попытки перехвата информации всегда можно детектировать по ошибкам на приёмной стороне.

\subsubsection{Невозможность достоверного различения неортогональных состояний}\label{no_discrimination_theorem}
Невозможность достоверного различения неортогональных квантовых состояний \cite{non_orthogonal_states_discrimination_theorem}~--- важный результат, на котором также во многом основывается секретность протоколов квантовой криптографии.

Этот результат можно сформулировать следующим образом: для чистых состояний \state{\psi_0} и \state{\psi_1} таких, что
$\scalar{\psi_0}{\psi_1} = \cos \alpha \neq 0$, не существует измерения $\{M_0, M_1\}$, которое давало бы точный результат, то есть соответствовало бы условиям
\begin{eqnarray}\label{measurement_operators_conditions}
  \conjstate{\psi_0} M_0 \state{\psi_0} = 1,\quad \conjstate{\psi_1} M_0 \state{\psi_1} = 0, \nonumber\\
  \conjstate{\psi_0} M_1 \state{\psi_0} = 0,\quad \conjstate{\psi_1} M_1 \state{\psi_1} = 1.
\end{eqnarray}

Докажем это утверждение. Допустим, такое измерение существует. Рассмотрим представление \state{\psi_1} как линейную комбинацию состояния \state{\psi_0} и его нормированного ортогонального дополнения \state{\psi_0^\bot}:
\begin{equation} \state{\psi_1} = a\state{\psi_0} + b\state{\psi_0^\bot},\quad |a^2| + |b^2| = 1. \end{equation}
Так как \state{\psi_0} и \state{\psi_1} неортогональны, то $0 < |a| < 1,\  0 < |b| < 1$.
Из условий (\ref{measurement_operators_conditions}) на операторы очевидно следует, что $\sqrt{M_1}\state{\psi_0} = 0$, а значит,
\begin{equation} \sqrt{M_1}\state{\psi_1} = \sqrt{M_1} a\state{\psi_0} + \sqrt{M_1} b\state{\psi_0^\bot} = 
\sqrt{M_1} b\state{\psi_0^\bot}, \end{equation}
из чего следует, что последнее равенство в (\ref{measurement_operators_conditions}) можно записать как
\begin{equation} \conjstate{\psi_1} M_1 \state{\psi_1} = |b^2| \conjstate{\psi_0^\bot} M_1 \state{\psi_0^\bot} \leq |b^2|, \end{equation}
что противоречит (\ref{measurement_operators_conditions}) в силу $|b| < 1$. Полученное противоречие доказывает невозможность различения неортогональных состояний.

\subsubsection{Чёткие и нечёткие наблюдаемые}
Обычно под наблюдаемой подразумевают только ортогональное разложение единицы. Такие наблюдаемые будем называть \textit{чёткими наблюдаемыми}\cite{holevo}. В то же время требование взаимной ортогональности всех операторов не является обязательным, а в некоторых случаях выгоднее пользоваться наблюдаемыми, в которых не все операторы ортогональны друг другу, в целях получения максимального количества информации. Такие наблюдаемые называются \textit{нечёткими}.

На первый взгляд нечёткие наблюдаемые просто смешивают вероятности разных исходов и не могут принести дополнительной пользы. Однако это не так.
Рассмотрим пример, как нечёткая наблюдаемая может помочь различить неортогональные состояния \state{\varphi} и \state{\psi}: $ \scalar{\varphi}{\psi} = \cos \eta \neq 0.$

Одно из возможных измерений для такой пары состояний принято называть <<измерение с тремя исходами>>, и оно использует три результата: $\{0, 1, ?\}$. Соответствующие эрмитовы операторы равны

\begin{eqnarray}\label{unambigious_measurement}
  M_0 = \frac{\state{\psi^\bot}\conjstate{\psi^\bot}}{1 + \cos \eta} = \frac{I - \state{\psi}\conjstate{\psi}}{1 + \cos \eta}, \nonumber\\
  M_1 = \frac{\state{\varphi^\bot}\conjstate{\varphi^\bot}}{1 + \cos \eta} = \frac{I - \state{\varphi}\conjstate{\varphi}}{1 + \cos \eta}, \\
  M_? = I - M_0 - M_1. \nonumber
\end{eqnarray}

Несложно обнаружить, что 
$$ \tr M_0 \state{\psi}\conjstate{\psi} = \conjstate{\psi} M_0 \state{\psi} = \frac{\scalar{\psi}{\psi^\bot} \scalar{\psi^\bot}{\psi}}{1 + \cos \eta} = 0, $$
и аналогично $\tr M_1 \state{\varphi}\conjstate{\varphi} = 0$. Это значит, что при применении такого измерения нет шансов получить исход 0 при измерении состояния \state{\psi}, а при измерении состояния \state{\varphi} не может получиться исход 1. Это означает, что такое измерение позволяет различать неортогональные состояния без ошибок. Цена этого~--- некоторая вероятность (равная $\cos \eta$) получить несовместный исход <<?>>, который соответствует уклонению от ответа.

\subsection{Составные квантовые системы}
Рассмотрение квантовых систем из нескольких частиц может привести к интересным свойствам, которые не встречаются в классическом случае. Еще в переписке Эйнштейна, Подольского и Розена \cite{epr_paradox} были отмечены необычные свойства составных квантовых систем, которые противоречили принципу локальности: получалось, что действия над одной подсистемой могут мгновенно оказывать влияние на другую подсистему вне зависимости от расстояния между ними. 
\subsubsection{Тензорное произведение}
Для начала определим, в каком пространстве находятся составные квантовые системы.

Рассмотрим наиболее простой случай двух кубитов. Интуитивно понятно, что возможны 4 варианта их совместного состояния:
\begin{itemize}
  \item оба кубита в состоянии \state{0};
  \item первый кубит в состоянии \state{0}, второй~-- в состоянии \state{1};
  \item первый кубит в состоянии \state{1}, второй~-- в состоянии \state{0};
  \item оба кубита в состоянии \state{1}.
\end{itemize}
Именно эти четыре вектора и будут являться базисными в пространстве двух кубитов.

Формально это описывается следующим образом. Если есть пространства $\mathcal{H}_1$ и $\mathcal{H}_2$ c размерностями $d_1$ и $d_2$ и ортонормированными базисами $\{e_i\}$ и $\{f_i\}$, то можно определить пространство с базисом $\{e_i \otimes f_j\},\, i = \overline{1,d_1},\, j = \overline{1,d_2}$. Если ввести на этом пространстве скалярное произведение
\begin{equation}
  \scalar{e_i \otimes f_j}{e_m \otimes f_n} = \scalar{e_i}{e_m} \cdot \scalar{f_j}{f_n} 
\end{equation}
 и продолжить его по линейности на остальные векторы, то в результате получим гильбертово пространство, называемое тензорным произведением $\mathcal{H}_1$ и $\mathcal{H}_2$, обозначаемое $\mathcal{H}_1 \otimes \mathcal{H}_2$.

Тензорное произведение операторов $A_1 \in \mathcal{S}(\mathcal{H}_1)$ и $A_2 \in \mathcal{S}(\mathcal{H}_2)$~--- оператор $A_1 \otimes A_2$ в пространстве $\mathcal{H}_1 \otimes \mathcal{H}_2$, который действует по закону
\begin{equation} 
  (A_1 \otimes A_2)\state{e_1 \otimes e_2} = (A_1\state{e_1})\otimes(A_2\state{e_2}). 
\end{equation}

Встает вопрос о том, всякое ли состояние в пространстве $\mathcal{H}_1 \otimes \mathcal{H}_2$ можно задать как тензорное произведение состояний из частичных пространств $\mathcal{H}_1$ и $\mathcal{H}_2$. Ответ на него отрицателен.
Классическим контрпримером является состояние в пространстве двух кубитов, называемое ЭПР:
\begin{equation}\label{epr_state}  
  \state{\psi_{EPR}} = \frac{\state{00} + \state{11}}{\sqrt{2}}.
\end{equation}
Легко видеть, что это состояние невозможно представить в виде тензорного произведения одночастичных состояний: 
\begin{equation}
  \state{\psi_{EPR}} \neq (a_1\state{0} + b_1\state{1}) \otimes (a_2 \state{0} + b_2 \state{1}).
\end{equation}

\subsubsection{Частичный оператор плотности и частичные измерения}
После определения тензорного произведения операторов плотности возникает необходимость определить обратную операцию, с помощью которой можно было бы по состоянию $\rho_1 \otimes \rho_2 \in \mathcal{H}_1 \otimes \mathcal{H}_2$ получить исходные операторы $\rho_1 \in \mathcal{H}_1$ и $\rho_2 \in \mathcal{H}_2$. 
Такая операция называется \textit{взятием частичного следа} и определяется следующим образом:
\begin{equation} \tr_{\mathcal{H}_2}\rho_{12} = \sum_{i,j,k} \state{e_i}\conjstate{e_j} \conjstate{e_i \otimes f_k} \rho_{12} \state{e_j \otimes f_k}. \end{equation}
Аналогично для частичного следа по первому подпространству:
\begin{equation} \tr_{\mathcal{H}_1}\rho_{12} = \sum_{i,j,k} \state{f_i}\conjstate{f_j} \conjstate{e_k \otimes f_i} \rho_{12} \state{e_k \otimes f_j}. \end{equation}

По определению этой операции видно, что:
\begin{eqnarray}
\tr_{\mathcal{H}_2} \rho_1 \otimes \rho_2 = \rho_1, \nonumber\\
\tr_{\mathcal{H}_1} \rho_1 \otimes \rho_2 = \rho_2.
\end{eqnarray}

Рассмотрим теперь ситуацию, когда квантовое состояние распределено между двумя участниками, один из которых производит измерение над своей подсистемой. Такое действие называют \textit{частичным измерением}.

При измерении одной подсистемы над второй не производится активных действий, поэтому в разложении единицы, описывающем общее измерение, все операторы, соответствующие второй подсистеме, будут тождественными. Например, если первый участник применяет измерение $\{\state{0}\conjstate{0}, \state{1}\conjstate{1}\}$, то в составной системе это измерение будет выглядеть так:
\begin{equation}M_0 = \state{0}\conjstate{0}_1 \otimes I_2,\qquad M_1 = \state{1}\conjstate{1}_1 \otimes I_2.\end{equation}
Стоит заметить, что несмотря на тождественные операторы в правой части, измерение первой подсистемы в общем случае \textit{влияет на состояние второй подсистемы}.

\subsubsection{Квантовая запутанность}
Квантовая запутанность~--- квантовомеханическое явление, при котором квантовые состояния двух или большего числа объектов оказываются взаимозависимыми. Такая взаимозависимость сохраняется, даже если эти объекты разнесены в пространстве за пределы любых известных взаимодействий, что находится в логическом противоречии с принципом локальности. 
Например, можно получить пару фотонов, находящихся в запутанном состоянии, и тогда если при измерении спина первой частицы спиральность оказывается положительной, то спиральность второй всегда оказывается отрицательной, и наоборот.

Рассмотрим состояние ЭПР (\ref{epr_state}) в пространстве двух кубитов
\begin{equation}\state{\psi_{EPR}} = \frac{\state{00} + \state{11}}{\sqrt{2}}\end{equation}
и посмотрим, что будет, если провести измерение над первой подсистемой. При выпадении исхода 0 начальное состояние перейдет в 
\begin{equation} \frac{\sqrt{M_0}\state{\psi_{EPR}}\conjstate{\psi_{EPR}}\sqrt{M_0}}{ \conjstate{\psi_{EPR}}M_0\state{\psi_{EPR}}} = \state{00}\conjstate{00}, \end{equation}
что соответствует чистому состоянию \state{00}. Аналогично при исходе 1 начальное состояние перейдет в \state{11}. Это говорит об удивительном факте: измерение одной части квантового состояния может изменять всё состояние в целом.

Это свойство имеет место не для произвольных квантовых состояний, а только для запутанных. Запутанные состояния определяются как состояния в составном пространстве, которые нельзя представить в виде тензорного произведения состояний в каждом из частичных пространств.

Для состояний, которые не являются запутанными, подобное свойство не имеет места: измерение одной подсистемы никак не влияет на состояние второй.

\subsubsection{Невозможность клонирования квантовых состояний}\label{no_cloning_theorem}
В квантовой криптографии важен еще один результат из теории составных квантовых систем. Выше было показано, что неортогональные квантовые состояния нельзя достоверно различить. Здесь будет показано, что такие состояния нельзя и клонировать \cite{no_cloning_theorem}~--- например, чтобы собрать более полную статистику результатов измерений.

Преобразование U, клонирующее произвольное чистое состояние \state{\psi}, можно описать так:
\begin{equation}\label{no_cloning_proof_definition}
  U\state{\psi}\otimes\state{A} = \state{\psi}\otimes\state{\psi},  
\end{equation}
где \state{A}~--- исходное состояние вспомогательной системы.

Чтобы показать невозможность такого преобразования, достаточно рассмотреть его действие на базисные состояния \state{0} и \state{1}:
\begin{eqnarray}\label{no_cloning_proof_bases}
  U\state{0}\otimes\state{A} = \state{0}\otimes\state{0}, \nonumber\\
  U\state{1}\otimes\state{A} = \state{1}\otimes\state{1},
\end{eqnarray}
а также на состояние $\frac{1}{\sqrt{2}}(\state{0} + \state{1})$. В силу линейности оператора $U$ и соотношений (\ref{no_cloning_proof_bases}) должно выполняться
\begin{equation}U(\frac{1}{\sqrt{2}}(\state{0} + \state{1}) \otimes \state{A}) = \frac{1}{\sqrt{2}}(\state{0} \otimes \state{0} + \state{1} \otimes \state{1}).\end{equation}
С другой стороны, по определению $U$ (\ref{no_cloning_proof_definition}) должно получаться
\begin{equation}U(\frac{1}{\sqrt{2}}(\state{0} + \state{1}) \otimes \state{A}) = \frac{1}{2}(\state{0} + \state{1}) \otimes (\state{0} + \state{1}).\end{equation}

Полученное противоречие доказывает невозможность клонирования произвольных квантовых состояний. Стоит отметить, что клонировать состояния из ортогонального набора можно: для этого достаточно их измерить и приготовить состояние, соответствующее результату измерения.