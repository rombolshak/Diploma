%%% Макет страницы %%%
\geometry{a4paper,top=2cm,bottom=2cm,left=2.5cm,right=1cm}

%%% Кодировки и шрифты %%%
\ifxetex
  \setmainlanguage{russian}
  \setotherlanguage{english}
  \defaultfontfeatures{Ligatures=TeX,Mapping=tex-text}
  \setmainfont{Times New Roman}
  \newfontfamily\cyrillicfont{Times New Roman}
  \setsansfont{Arial}
  \newfontfamily\cyrillicfontsf{Arial}
  \setmonofont{Courier New}
  \newfontfamily\cyrillicfonttt{Courier New}
\else
  \IfFileExists{pscyr.sty}{\renewcommand{\rmdefault}{ftm}}{}
\fi
%\frenchspacing

%%% Интервалы %%%
\linespread{1.3}                    % Полуторный интвервал (ГОСТ Р 7.0.11-2011, 5.3.6)
%\onehalfspacing

%%% Выравнивание и переносы %%%
\sloppy					% Избавляемся от переполнений
\clubpenalty=10000		% Запрещаем разрыв страницы после первой строки абзаца
\widowpenalty=10000		% Запрещаем разрыв страницы после последней строки абзаца

%%% Библиография %%%
\makeatletter
\bibliographystyle{utf8gost705u}	% Оформляем библиографию в соответствии с ГОСТ 7.0.5
\renewcommand{\@biblabel}[1]{#1.}	% Заменяем библиографию с квадратных скобок на точку:
\makeatother

%%% Изображения %%%
\graphicspath{{images/}} % Пути к изображениям

%%% Цвета гиперссылок %%%
\definecolor{linkcolor}{rgb}{0.9,0,0}
\definecolor{citecolor}{rgb}{0,0.6,0}
\definecolor{urlcolor}{rgb}{0,0,1}
\hypersetup{
    colorlinks, linkcolor={linkcolor},
    citecolor={citecolor}, urlcolor={urlcolor}
}

%%% \paragraph работает как \subsubsubsection если установить 4
\setcounter{secnumdepth}{3}

\titleformat{\paragraph}
{\normalfont\normalsize\bfseries}{\theparagraph}{1em}{}
\titlespacing*{\paragraph}
{0pt}{3.25ex plus 1ex minus .2ex}{1.5ex plus .2ex}


%%% Оглавление %%%
\renewcommand\cftchapdotsep{\cftdotsep}
\renewcommand\cftchapleader{\cftdotfill{\cftchapdotsep}}

%\renewcommand{\cfttoctitlefont}{\hspace{0.38\textwidth} \bfseries\MakeUppercase}
%\renewcommand{\cftbeforetoctitleskip}{-1em}
%\renewcommand{\cftaftertoctitle}{\mbox{}\hfill \\ \mbox{}\hfill{\footnotesize Стр.}\vspace{-2.5em}}
%\renewcommand{\cftsecfont}{\hspace{31pt}}
%\renewcommand{\cftsubsecfont}{\hspace{11pt}}
%\renewcommand{\cftbeforechapskip}{1em}
%\renewcommand{\cftparskip}{-1mm}
%\renewcommand{\cftdotsep}{1}
\setcounter{tocdepth}{2}
\renewcommand{\cftchapfont}{\normalsize\bfseries \MakeUppercase{Глава} }

%% новая секция для специальных разделов likechapter %%	 
\makeatletter
    \renewcommand{\@dotsep}{2}
    \newcommand{\l@likechapter}[2]{{\bfseries\@dottedtocline{0}{0pt}{0pt}{\MakeUppercase{#1}}{#2}}}
\makeatother


%%% Списки %%%
% Используем дефис для ненумерованных списков (ГОСТ 2.105-95, 4.1.7)
\makeatletter
    \AddEnumerateCounter{\asbuk}{\@asbuk}{м)}
\makeatother
\setlist{nolistsep}
\renewcommand{\labelitemi}{\normalfont\bfseries{--}} 
\renewcommand{\labelenumii}{\asbuk{enumii})}
\renewcommand{\labelenumi}{\arabic{enumi})}

%%% Колонтитулы %%%
% Порядковый номер страницы печатают на середине верхнего поля страницы (ГОСТ Р 7.0.11-2011, 5.3.8)
\makeatletter
\let\ps@plain\ps@fancy              % Подчиняем первые страницы каждой главы общим правилам
\makeatother
\pagestyle{fancy}                   % Меняем стиль оформления страниц
\fancyhf{}                          % Очищаем текущие значения
\fancyhead[C]{\thepage}             % Печатаем номер страницы на середине верхнего поля
\fancyheadoffset{0mm}
\fancyfootoffset{0mm}
\setlength{\headheight}{17pt}
\renewcommand{\footrulewidth}{0pt}
\renewcommand{\headrulewidth}{0pt}  % Убираем разделительную линию

