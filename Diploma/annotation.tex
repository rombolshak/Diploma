\begin{abstract}
  \thispagestyle{plain}
  \setcounter{page}{2}

Предметом изучения данной дипломной работы является релятивистский протокол квантого распределения ключей и протокол коррекции ошибок каскадным методом.
Целью работы является создание программного комплекса, моделирующего и визуализирующего процессы, происходящие в указанных протоколах. 

Дипломная работа содержит 90 страниц машинопечатного текста, 52 изображения, 1 таблицу, 61 формулу.
Состоит из введения, трех глав, заключения и списка литературных источников.

Во введении описывается актуальность и ставится цель работы.
Первая глава предоставляет теоретические знания, необходимые для понимания принципов квантового распределения ключей.
Вторая глава посвящена детальному разбору и анализу \begin{inparaenum}[\itshape 1\upshape)]
                                                      \item релятивистского протокола распределения ключей,
                                                      \item каскадному протоколу коррекции ошибок.
                                                    \end{inparaenum}
Третья глава описывает детали практической реализации полученного программного комплекса.
В заключении представлены выводы по теме исследования.

\end{abstract}
