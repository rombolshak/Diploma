\section{Ограничения на канал связи}
Идеальный одномодовый канал связи в открытом пространстве описывается параксиальным волновым уравнением с решением в форме Гауссова пучка. 
Дифракция пучка ограничивает максимальную длину линии передачи, которая становится зависящей от диаметра линз. 
В типичной симметричной конфигурации длина канала
\begin{equation*}
  L = \frac{2\pi\omega^2}{\lambda} \frac{\omega_0}{\omega} \sqrt{1 - \frac{\omega_0}{\omega}^2},
\end{equation*}
где $\omega_0$~--- перетяжка пучка, и $\omega$~--- радиус пучка в линзах.
Длина принимает максимальное значение в $\omega_0 / \omega = 1 / sqrt{2}$. В такой конфигурации
\begin{equation*}
  L = \frac{\pi\omega^2}{\lambda} = 2 \frac{\pi\omwga^2_0}{\lambda} = 2z_R,
\end{equation*}
где $z_R$~--- длина Рэлея.

Например, для 