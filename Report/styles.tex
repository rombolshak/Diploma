%%% Макет страницы %%%
\geometry{a4paper,top=2cm,bottom=2cm,left=3cm,right=1cm}

%%% Кодировки и шрифты %%%
\renewcommand{\rmdefault}{ftm} % Включаем Times New Roman

%%% Выравнивание и переносы %%%
\sloppy					% Избавляемся от переполнений
\clubpenalty=10000		% Запрещаем разрыв страницы после первой строки абзаца
\widowpenalty=10000		% Запрещаем разрыв страницы после последней строки абзаца

%%% Библиография %%%
\makeatletter
\bibliographystyle{utf8gost705u}	% Оформляем библиографию в соответствии с ГОСТ 7.0.5
\renewcommand{\@biblabel}[1]{#1.}	% Заменяем библиографию с квадратных скобок на точку:
\makeatother

%%% Изображения %%%
\graphicspath{{images/}} % Пути к изображениям

%%% Цвета гиперссылок %%%
\definecolor{linkcolor}{rgb}{0.9,0,0}
\definecolor{citecolor}{rgb}{0,0.6,0}
\definecolor{urlcolor}{rgb}{0,0,1}
\hypersetup{
    colorlinks, linkcolor={linkcolor},
    citecolor={citecolor}, urlcolor={urlcolor}
}

\newtheorem{theorem}{Теорема}
\newtheorem{definition}{Определение}

%%% \paragraph работает как \subsubsubsection
\setcounter{secnumdepth}{4}

\titleformat{\paragraph}
{\normalfont\normalsize\bfseries}{\theparagraph}{1em}{}
\titlespacing*{\paragraph}
{0pt}{3.25ex plus 1ex minus .2ex}{1.5ex plus .2ex}


\newcommand{\state}[1]{\ensuremath{\left| #1 \right\rangle}}
\newcommand{\conjstate}[1]{\ensuremath{\left\langle #1 \right|}}
\newcommand{\scalar}[2]{\ensuremath{\left\langle #1|#2\right\rangle}}
\DeclareMathOperator{\tr}{Tr}
\DeclareMathOperator{\pr}{Pr}

%%% Оглавление %%%
%\renewcommand{\cftchapdotsep}{\cftdotsep}
