\section[Основные особенности протокола]{Основные особенности протокола релятивистского квантового распределения ключей}
Обычная, нерелятивистская, квантовая криптография основывается на фундаментальных принципах квантовой механики, однако, не привязана к элементарным частицам или другим физическим обхектам, которые несут передаваемые квантовые состояния. В описываемом протоколе \textit{релятивистской} квантовой криптографии это должна быть безмассовая частица, движущаяся со скоростью света, например, фотон. 
Это имеет значение, если принять во внимание пространственно-временную структуру коммуникации в пространстве Минковского, обращающейся к невозможности передать информацию быстрее, чем со скоростью света.
Эта явная связь с пространством-временем совершенно игнорируется в обычных протоколах квантового распределенеия ключей.

Описываемый протокол основан на протяженных по времени когерентных квантовых состояниях. Благодаря их протяженной природе, проведенная атака <<прием-перепосыл>> неизбежно повлечет детектируемые задержки.
Таким образом, детектирование действий подслушивателя может быть проведено учетом как ошибок, так и задержек сигнала. 
Это автоматически делает протокол полностью невосприимчивым к произвольным большим потерям в квантовом канале связи и создает потенциал для его использования в системах земля-спутник для создания глобального сервиса распределения ключей.

Так как задержки сигнала играют критическую роль в релятивистском подходе, протокол жизнеспособен только в каналах связи на открытом пространстве, расположенных по линии взгляда, где не существует более короткого пути меж двух сторон, и сигнал распространяется со скоростью света. 
Важно заметить, что протокол терпим к наличия воздуха на пути света, что немного задерживает сигнал по сравнению со скоростью света; это приводит лишь к необходимости достаточной протяженности по времени передаваемых квантовых состояний. 
В типичных условиях необходима протяженность примерно на 1 нс на каждый километр в воздухе.

Несмотря на то, что отслеживание точного времени требует, в общем случае, внешней синхронизации часов, описываемый протокол берет заботу о синхронизации между Алисой и Бобом на себя, никаких других схем внешней синхронизации не нужно. 
В то же время протоколу необходимо априорное знание расстояния между сторонами коммуникации, что требуется, например, если подслушиватель Ева пытается замедлить любую передачу между Алисой и Бобом.

Итак, основные особенности протокола включают в себя: 
\begin{inparaenum}[\itshape 1\upshape)]
\item протяженность квантовых состояний не обязана быть столь же большой, какова длина канала связи; она только лишь должна компенсировать задержки в канале относительно идеального канала со скоростью передачи равной скорости света в вакууме;
\item так как релитявистские принципы позволяют провести синхронизацию часов, внешние схемы синхронизации не требуются;
\item протокол предоставляет безусловную секретность ключа даже при использовании обычных слабых лазерных импульсов при сколь угодно больших потерях в канале связи; практические ограничения на потери определяются только темновыми шумами используемого детектора.
\end{inparaenum}
\clearpage