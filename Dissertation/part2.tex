\chapter{Длинное название главы, в которой мы смотрим на примеры того, как будут верстаться изображения и списки} \label{chapt2}

\section{Одиночное изображение} \label{sect2_1}

\begin{figure} [h] 
  \center
  \includegraphics [scale=0.27] {latex}
  \caption{TeX.} 
  \label{img:latex}  
\end{figure}

%\newpage
%============================================================================================================================
\section{Длинное название параграфа, в котором мы узнаём как сделать две картинки с общим номером и названием} \label{sect2_2}

А это две картинки под общим номером и названием:
\begin{figure}[h]
  \begin{minipage}[h]{0.49\linewidth}
    \center{\includegraphics[width=0.5\linewidth]{knuth1} \\ а)}
  \end{minipage}
  \hfill
  \begin{minipage}[h]{0.49\linewidth}
    \center{\includegraphics[width=0.5\linewidth]{knuth2} \\ б)}
  \end{minipage}
  \caption{Очень длинная подпись к изображению, на котором представлены две фотографии Дональда Кнута}
  \label{img:knuth}  
\end{figure}

%\newpage
%============================================================================================================================
\section{Пример вёрстки списков} \label{sect2_3}

\noindent Нумерованный список:
\begin{enumerate}
  \item Первый пункт.
  \item Второй пункт.
  \item Третий пункт.
\end{enumerate}

\noindent Маркированный список:
\begin{itemize}
  \item Первый пункт.
  \item Второй пункт.
  \item Третий пункт.
\end{itemize}

\noindent Вложенные списки:
\begin{itemize}
  \item Имеется маркированный список.
  \begin{enumerate}
    \item В нём лежит нумерованный список,
    \item в котором
    \begin{itemize}
      \item лежит ещё один маркированный список.
    \end{itemize}    
  \end{enumerate}
\end{itemize}


\clearpage
