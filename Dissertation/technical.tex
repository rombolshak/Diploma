\section[Технические подробности]{Технические подробности приготовления и измерения состояний}
Алиса активирует лазер в определенныый момент времени и получает на выходе интенсивное когерентное состояние, локализованное в интервале $l_{pac}$. 
При прохождении через интерферометр локализованное состояние преобразуется в состояние из двух половинок, разделенных интервалом $l$: $|\alpha_c\rangle_1 \otimes |\alpha_c\rangle_2$. 
Затем состояние через линзовую систему направляется в канал связи. Приготовление протяженного состояния из локализованного требует конечного времени.

На приемной стороне Боба классическое состояние вводится в волоконную часть. 
Через светоделитель состояние поступает на классический детектор, по импульсу тока на котором оценивается интенсивность состояния и записывается его время прилета. 
Затем сигнал отражается от фарадеевского зеркала, в зависимости от сигнала на детекторе ослабляется и становится равным $|\alpha\rangle_1 \otimes |\alpha\rangle_2$.
При прохождении второй половинки ослабленного состояния через фазовый модулятор на последний подается импульс напряжения и <<навешивается>> относительная фаза.
Получившееся состояние $|\alpha\rangle_1 \otimes |e^{i\varphi_B}\alpha\rangle_2$ направляется к Алисе.

Поскольку Алиса знает время приготовления своего состояния и расстояние $L$ между передающей и приемной станциями, при обратном проходе она активирует фазовый модулятор в момент прохождения первой половины состояния по нижнему, более длинному, плечу интерферометра. 
Из-за разности хода на втором светоделителе интерферируют передняя, из нижнего плеча, и задняя, из верхнего плеча интерферометра, половинки.
Таблица \ref{} отражает последовательное преобразование состояний по оптическому тракту.

На входе лавинного фотодетектора в центральном временно окне 2 состояние равно $|\frac{(e^{i\varphi_B} - e^{i\varphi_A})\alpha}{2}\rangle_2$.
При обратном проходе состояния в плече лазера являются холостыми.

Далее, если Боб выбрал $\varphi_B = \varphi_0$, а Алиса выбрала также $\varphi_A = \varphi_0$ (или $\varphi_B = \varphi_1$ и $\varphi_A = \varphi_1$), то отсчета в детекторе не будет из-за деструктивной интерференции.
В противоложном случае, когда Боб выбрал $\varphi_B = \varphi_0$, а Алиса выбрала $\varphi_A = \varphi_1$ (и аналогично $\varphi_B = \varphi_1$ и $\varphi_A = \varphi_0$), будет отсчет.
Таким образом, Алиса по отсчету детектора знает бит, выбранный Бобом.

Следует отметить, что данная схема является реализацией двух измерений, которые Алиса выбирает случайно путем выбора фазы. 
Фактически данное измерение реализует проекцию на состояние $|e^{i\varphi_B}\alpha\rangle_2 ~ {}_2 \langle e^{i\varphi_B}\alpha|$ и на его ортогональное дополнение $I - |e^{i\varphi_B}\alpha\rangle_2 ~ {}_2 \langle e^{i\varphi_B}\alpha|$.
\clearpage