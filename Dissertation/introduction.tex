\section*{Введение}				% Заголовок
\addcontentsline{toc}{section}{Введение}	% Добавляем его в оглавление

Квантовое распределение ключей (QKD)~--- концепт секретного распределения ключей, основанный на фундаментальных законах квантовой механики.
Квантовая криптография \cite{bennett1984QuacryPubkeydiscoitos, ekert1991QuacrybasBelthe, gisin2002Quacry, scarani2009secpraquakeydis, hughes2011Refquacry, lam2013QuacryConimp} приобрела популярность за обещание абсолютной секретности против подслушивания. <<Абсолютной>> понимается в том смысле, что секретность гарантирована
фундаментальными запретами квантовой механики (на копирование неизвестного квантового состояния и невозможности достоверной различимости неортогональных квантовых состояний) \cite{bennett1984QuacryPubkeydiscoitos, bennett1992Quacryusianytwononsta, wootters1982sinquacanbeclo, dieks1982CombyEPRdev}, 
а не нашими технологическими возможностями. Достоверная неразличимость неортогональных квантовых состояний приводит к тому, что любые попытки вторжения в канал связи с целью получения информации о передаваемых состояниях вызывают их неизбежное возмущение, что ведет к ошибкам на приемной стороне и детектированию подслушивателя.
Если ошибка на приемной стороне не превосходит некоторой критической величины\footnote{Величина критической ошибки определяется конкретным протоколом}, то ошибки могут быть исправлены через аутентичный открытый классический канал связи. В результате последующего сжатия (хеширования \cite{bennett1995Genpriamp}) очищенного ключа возникает секретный ключ, известный только двум легитимным пользователям.

Однако, \textit{практические} схемы реализации QKD~--- серьезный вызов для ученых, так как все реализации так или иначе отличаются от теоретических моделей. 
Две основные проблемы всех существующих реализаций, ни одна из которых не может быть эффективно устранена: 
\begin{inparaenum}[\itshape 1\upshape)]
\item любой существующий в настоящее время источник фотонов имеет ненулевую вероятность испустить два или более фотонов одновременно, в то время как в теории нужен ровно один \cite{lounis2000Sinphodemsinmolrootem, benjamin2000Sinphodem}, и
\item наличие потерь в квантовом канале связи.
\end{inparaenum}

В реальной ситуации неоднофотонность источника вместе с потерями в квантовом канале связи приводит к тому, что все базовые протоколы распредения ключей: BB84, B92, SARG04, decoy-state (с состояниями-ловушками), phase-time (фазово-временное кодирование) оказываются неустойчивыми относительно PNS атаки (атака с расщеплением по числу фотонов) и не гарантируют секретность ключей, если длина квантового канала связи превышает некоторую критическую величину. 

Протоколы используются как в оптоволоконных системах квантовой криптографии, так и в системах, работающих через открытое пространство.
Конечной целью работ по квантовой криптографии в открытом пространстве является создание глобальной системы распределения ключей на большие расстояния через низкоорбитальные спутники. 
При передаче ключей через открытое пространство могут быть использованы протоколы, стойкость которых базируется на запретах только квантовой механики, применяемые в оптоволоконных системах квантовой криптографии. 
Однако при не строго однофотонном источнике квантовых состояний и потерях в канале связи дальность передачи секретных ключей при помощи таких протоколов ограничена \cite{scarani2009secpraquakeydis}. 
В принципе можно сформулировать протоколы, дальность которых не ограничена, но при этом неизбежно требуются априорное знание величины потерь и их контроль в канале связи. 
Если для оптоволоконных систем такой подход может оказаться достаточным, то для открытого пространства он неприемлем, посколько априорно потери в канале связи неизвестны и могут меняться в течение передачи ключей. По-видимому, при неоднофотонном источнике и больших априорно не известных потерях, одних только фундаментальных запретов квантовой механики недостаточно для формулировки протоколов, гарантирующих секретность ключей.

Возникает принципиальный и практически важный вопрос о том, существуют ли протоколы квантового распределения ключей, которые обеспечивают безусловную секретность ключей при не строго 
однофотонном источнике и произвольных потерях в квантовом канале связи. Ниже будет предъявлен такой протокол. Данный протокол, кроме ограничений квантовой механики на различимость квантовых состояний, использует дополнительные ограничения, диктуемые специальной теорией относительности.
\clearpage