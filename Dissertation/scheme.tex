\section[Общая схема протокола]{Общая идея релятивистского квантового распределения ключей через открытое пространство без синхронизации часов}
\begin{itemize}
  \item Алиса и Боб контролируют области пространства, необходимые для приготовления и измерения протяженных квантовых состояний.
  \item Расстояние $L$ между Алисой и Бобом всем известно и являетс параметром протокола. Алиса и Боб имеют часы, но не имеют общего начала отсчета времени (часы не синхронизированы).
  \item Происходит передача серии состояний Алисой. Каждая посылка происходит в случайный момент времени внутри интервала $\Delta T$. 
  Достаточно, чтобы Алиса случайны выбирала один из двух моментов посылки сигнала внутри интервала $\Delta T$. 
  Алиса готовит протяженное классическое состояние, состоящее из пары интенсивных когерентных пакетов, разделенных интервалом $l > l_{pac}$ ($l_{pac}$~--- ширина пакета, см. ниже): 
  $|\alpha_c\rangle_1 \otimes|\alpha_c\rangle_2$ (индексы <<1>> и <<2>> отвечают пакетам, локализованным в моменты времени 1 и 2); среднее число фотонов в состоянии $\mu_c = |\alpha_c|^2 \gg 1$. 
  Временное разрешение проводится с точностью до ширины пакета $l_{pac}$ (интервалы времени, меньшие $l_{pac} / c$, считаются нулевыми). 
  Момент времени $t_{A,i}$ посылки состояния в канал связи Алисой фиксируется по своим часам.
  
  \item Аппаратура Боба на приемной стороне работает в ждущем режиме. При помощи быстрого классического детектора Боб фиксирует момент прихода состояния в каждой $i$-й посылке $t_{B,i}$. 
  Далее классический сигнал ослабляется до квазиоднофотонного уровня, и при помощи фазового модулятора на одну из <<половинок>> (заднюю) случайным образом навешивается фаза. 
  Состояние $|\alpha\rangle_1 \otimes |e^{i\varphi_B}\alpha\rangle_2$ ($\mu = |\alpha|^2 < 1$) направляется обратно к Алисе 
  \footnote{Все задержки на стороне Боба, связанные с обработкой, заранее известны. Их величина непринципиальна и считается включенной в моменты $t_{A,B,i}$ и $t'_{A,B,i}$.}. 
  Значение относительной фазы у двух импульсов $\varphi_B = \varphi_0$ отвечает выбору логического 0 в ключе, а $\varphi_B = \varphi_1$~--- логической 1. 
  Кодирование осуществляется на стороне Боба.
  
  \item Алиса, зная расстояние $L$ и время отправки $t_{A,i}$ по своим часам своего состояния в канал связи, знает время прихода квантового состояния от Боба $t'_{A,i}$, преобразует состояния, случайно и независимо от Боба изменяет относительную фазу одной из <<половинок>>: 
  $|\alpha\rangle_1 \otimes |e^{i\varphi_B}\alpha\rangle_2 \rightarrow |\frac{\alpha}{2}\rangle_1 \otimes |\frac{(e^{i\varphi_B} - e^{i\varphi_A})\alpha}{2}\rangle_2$
  ($\varphi_A = \varphi_0$ или $\varphi_A = \varphi_1$), и производит измерения \textit{только в определенном временном окне}. 
  Если $\varphi_A \neq \varphi_B$, то возникает отсчет в детекторе, а если $\varphi_A = \varphi_B$, то отсчета не возникает. В результате Алиса знает, какой бит ключа посылал Боб.
  
  \item После проведения серии посылок Боб сообщает Алисе интервалы времени мужде сосендними посылками, которые он фиксировал по своим часам. Алиса сравнивает их со своими интервалами времени между посылками по своим часам. Подсчитывается доля их несовпадений $\eta$. Соседние посылки, интервалы между которыми не совпали, Алиса и Боб отбрасывают.
  
  \item Далее часть последовательности Алисой и Бобом раскрывается и сравнивается для оценки вероятности ошибки. Если ошибка меньше критической, то происходит исправление ошибок через открытый классчисеский канал связи. Затем происходит сжатие очищенного ключа. В результате возникает секретный ключ, известный только Алисе и Бобу.
\end{itemize}

Отметим, что Алиса и Боб не должны следить за средним числом долетевших посылок. Потери в канале связи, как будет показано позднее, вообще не входят в критерий секретности ключей.
